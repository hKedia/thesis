\chapter{Introduction}\label{chapter::introduction}

\section{Problem Context}
Blockchain technology emerged in 2008 with the creation of Bitcoin, a decentralized protocol for exchanging value among peers on the internet. With the Bitcoin network, it became possible to send value across the internet without any 3rd party.

Soon later, in 2014, Ethereum was invented. It allowed us to create complex applications by writing programs in a Turing complete language. These programs called smart contracts run as they are written and once they are deployed on the blockchain, they become immutable.

Blockchain, the technology which enabled Bitcoin and Ethereum, also enabled the emergence of decentralized applications. But currently, it's not clear, what a decentralized app or dApp is? As of this writing, the most popular platform for building dApps is Ethereum. It uses solidity as it's smart contracting language. Applications built on Ethereum uses a combination of smart contracts along with a traditional web architecture. The front end of the application talks to the smart contract for interacting with the blockchain and uses traditional storage for handling large data sets.

But, do dApps need a smart contract? Is it possible to create dApps without smart contracts? Also, on what specific use cases are smart contracts required?

To explore these questions we created a decentralized file sharing dApp both on Ethereum, a 1 layered protocol and Blockstack, a 2 layered protocol. 

\section{Thesis Statement}
In this thesis, we want to explore what constitutes a decentralized application and how it differs from a traditional web application. We will also explore smart contracts, their security aspects, and certain use cases where they are required as part of a decentralized application.

Based on the below metrics, we will analyze our dApp build on Ethereum and Blockstack.
\begin{itemize}
	\item User Experience
	\item Scalability
	\item Security
\end{itemize}

Above analysis will allow us to explore questions related to smart contract security and application scalability. Results from this analysis can help us determine what constitutes a secure smart contract platform?

At the end of this thesis, we will have a clear understanding of decentralized applications, when using smart contracts makes sense and how to make secure scalable dApps.



How the thesis will go? Each chapter.
