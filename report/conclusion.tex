\chapter{Conclusion \& Outlook}\label{chapter::conclusion}
	With the rapidly changing technologies that enables the Information Economy, its important that individuals own their data. A Self-Sovereign identity is the most important concept that Blockchain technology enables. A Decentralized Public Key Infrastructure (DPKI) built on top of a Blockchain abstracts all the complexity associated with key management and enables everyone to own their digital identity. Building applications around the concept of user owned identities enable data ownership by giving control of data to the users who create them.
	
	This thesis explored the different concepts which make up a web application namely Data, Identity, Value, Computation and Bandwidth; and described how each concept can be re-imagined using Blockchain and other peer-to-peer protocols.
	
	We analyzed the state-of-the-art by building proof-of-concept applications using two of the most popular decentralized applications platform, Ethereum and Blockstack. 
	
	Ethereum, being a "layer 1" system in its current state, handles all the complexity and the computations at the blockchain layer. This makes scaling and upgrading Ethereum a relatively complex task. Moreover, the smart contract language of Ethereum, Solidity, is relatively new and contains several design flaws and is prone to various attacks.
	
	Blockstack, being a "layer 2" system puts minimal logic at the blockchain layer and handles the complexity and computations off-chain. The design principles of Blockstack makes it relatively easy to scale and upgrade. Smart contracts on Blockstack are Turing incomplete and written in Clarity, a list processing (LISP) language.
	
	It's common notion that decentralized applications require smart contracts, however, we demonstrate in this thesis that it's possible to build such applications without a smart contract. In fact, smart contracts are required only in certain use cases, for example, an Escrow application.