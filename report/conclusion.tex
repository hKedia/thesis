\chapter{Conclusion \& Outlook}\label{chapter::conclusion}
With the rapidly changing technologies that enable the Information Economy, individuals must own their data. A Self-Sovereign identity is the most important concept that Blockchain technology enables. A Decentralized Public Key Infrastructure (DPKI) built on top of a Blockchain abstract all the complexity associated with key management and enables everyone to own their digital identity. Building applications around the concept of user-owned identities enable data ownership by giving control of data to the users who create them.

This thesis explored the different concepts which make up a web application, namely Data, Identity, Value, Computation, and Bandwidth, and described how each concept could be re-imagined using Blockchain and other peer-to-peer protocols.

We analyzed the state-of-the-art by building proof-of-concept applications using two of the most popular decentralized applications platform, Ethereum, and Blockstack. 

Ethereum, being a ``layer 1" system in its current state, handles all the complexity and the computations at the blockchain layer, this makes scaling and upgrading Ethereum a relatively complex task. Moreover, the smart contract language of Ethereum, Solidity, is relatively new and contains several design flaws and is prone to various attacks.

Blockstack, being a ``layer 2" system, puts minimal logic at the blockchain layer and handles the complexity and computations off-chain. The design principles of Blockstack makes it relatively easy to scale and upgrade. Smart contracts on Blockstack are Turing incomplete and written in Clarity, a list processing (LISP) language.

It is a common notion that decentralized applications require smart contracts; however, we demonstrate in this thesis that it is possible to build such applications without a smart contract. In fact, smart contracts are required only in certain use cases, for example, an Escrow application.

Public Blockchains will power the next generation financial systems and will act as a Value layer in the Internet protocol stack; therefore, they must be constructed with strong security guarantees. The code should be formally verified, and there should be a mechanism for people using the system to agree on matters like who pays and who decides. A Governance system solves the problem of who decides, and a Treasury system solves the problem of who pays.

It is also clear that the \textit{proof-of-work} consensus algorithm is energy-intensive and not sustainable in the longer run. Blockchains using proof-of-work will either upgrade or will be replaced by newer protocols based on more sustainable consensus algorithms like \textit{proof-of-stake}. There has been extensive research during the past years in proof-of-stake algorithms; Ouroboros\cite{kiayias2017ouroboros}, Snow White\cite{daian2017snow} and Algorand\cite{gilad2017algorand} are few of the proof-of-stake consensus algorithms that are provably secure with strong security guarantees.

Combining all the concepts and technologies introduced in the thesis, we can imagine how the web of the future might look like. Devices with a secure enclave enable secure storage of private keys. The corresponding public keys can be registered with a username on a DPKI system anchored on a Blockchain. Multiple devices can be connected using mesh networking where routing is done using IPv6. Software-defined networks (SDNs) can be employed for network management separating data and control layer, and providing each device connected to the network with a unique identifier. The broader internet can be connected with other mesh networks with a decentralized bandwidth market serving as an exchange point. Similarly, various computations can be spread across multiple devices with Blockchain serving as a unit of account for a decentralized computation market.

Technologies that will enable a truly decentralized and open web is in rapid development and is continuously changing. It will be interesting to see how they evolve.