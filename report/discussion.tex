\chapter{Discussion}\label{chapter::discussion}

	Chapter~\ref{chapter::results} analyzed and compared the two \textit{proof-of-concept} applications and their underlying technology. This chapter discusses the results.
	
	\section{On Decentralized Application Platform}
	Analyzing the underlying platforms on which the two applications are built, it is clear that Blockstack, due to its layered architecture is much more flexible and robust than Ethereum. Blockstack's separation of computations and storage from the blockchain layer makes it easier to scale and upgrade. It's integrated DPKI system offers its users with self-sovereign identity. Building applications around the concept of a decentralized identity enable us to build a web where everyone owns their data.
	
	Ethereum, too, is moving towards a layered architecture with its Eth 2.0 upgrade. The upgrade is spread across three phases. The complete specifications for the upgrade can be found at \url{https://github.com/ethereum/eth2.0-specs}.
	
	\section{On Decentralized Storage}
	Applications with data ownership involve encrypted data where storage systems are incentivized to store data. IPFS (see Section~\ref{sec:ipfs}) offers a content-addressed storage systems using p2p technologies like DHT, Git and BitTorrent. Gaia (see Section~\ref{sec:blockstack-gaia}) offers mutable data storage using existing cloud infrastructure. Both IPFS and Gaia enables a decentralized storage system serving different use cases. IPFS is more suitable for applications where the nature of data is immutable and where decentralization precedes the performance. Gaia, on the other hand, is more suitable for applications where is data is mutable, and data availability is crucial.
	
	\section{On Blockchains}
	Blockchain is the key enabler for building decentralized applications. It serves as the layer for value transfer in the Internet protocol stack. As inferred from results, blockchains based on \textit{proof-of-stake} are much more scalable and computationally inexpensive to use than blockchains based on \textit{proof-of-work}. Since Blockchain enables a peer-to-peer information economy, it essential that it is built with strong security guarantees. Also, as public blockchains do not have a single entity controlling the system, a blockchain needs to have a treasury and a governance system.
	