\begin{abstract}
	Humans have evolved over thousands of years building systems which deals with land ownership and property rights. With the advent of Internet our lives has become more and more digital, but we have no experience in managing data ownership. It's clear that data is becoming the new currency in today's digital economy. Big tech companies understood this a long time ago and therefore offered their services free of charge in exchange of our data which they then used to generate profits, control our perception about how we see the world and also tamper with public affairs like the election. There's clearly a need to define data ownership and build systems which enable users to own their data.
	
	With data ownership comes the question of digital identity. How can we identify ourselves over the internet? With username and passwords we can uniquely identify ourselves when using a service, but then we have to create an identity for each service we want to use. It has another drawback, i.e. our passwords are stored on a central server which is prone to hacking. There exists systems like \textit{Google Sign-in} or \textit{Facebook Connect} which allows us to carry our identity across multiple services but then again this identity is not owned by the user but by Google or Facebook. Therefore, there is a need for a self-sovereign identity which is owned by the user and can be verified independently by anyone.
	
	To define a model for Data Ownership, lets looks at Land Ownership. In a land ownership model, at any given point in time, a property has a fixed Geo-location while the owner can be anywhere in the Geo-space. Conversely, In a data ownership model, at any given point in time, a user has a fixed identity while the data can be anywhere on the internet.
	
	Blockchain along with Public key cryptography allows us to build a Decentralized Public Key Infrastructure (DPKI) thereby empowering users to create self-sovereign identity. Combining self-sovereign identity with encrypted storage enable us to build systems where users own their identity as well as their data.
	
	Blockchain also enables Smart Contracts, which are self executing code and which run when some conditions are met without requiring any intervention by any third party.
	
	Blockchain along with smart contracts and decentralized identity allows us to build interesting applications which are more aligned with the ethos of how the Web was intended in the first place.
	
	This thesis explores technologies like smart contracts and decentralized identity and identifies the properties of a secure decentralized application.
\end{abstract}